\documentclass[12pt]{article}
\renewcommand{\baselinestretch}{1.2}
\usepackage[utf8]{vietnam}
\usepackage[a4paper, total={6in, 8in}]{geometry}
\usepackage[english]{babel}
\usepackage{hyperref}
\usepackage{mathtools}
\usepackage{amssymb}
\usepackage{indentfirst}
\usepackage{graphicx}
\usepackage{minted}
\usepackage{ragged2e}
\usepackage{multirow}
\usepackage{subcaption}
\usepackage{xurl}
\usepackage{amsmath}
\usepackage{makecell}
\usepackage{algorithm}
\usepackage{algpseudocode}
\renewcommand\theadalign{bc}
\renewcommand\theadfont{\bfseries}
\renewcommand\theadgape{\Gape[4pt]}
\renewcommand\cellgape{\Gape[4pt]}
\usepackage{pbox}
\usepackage{graphicx}
\usepackage{diagbox}
\usepackage{listings}
\usepackage{xcolor}

\hypersetup{
    colorlinks=true,
    linkcolor=blue,
    citecolor=blue,
    urlcolor=blue,
}
\lstset{
    backgroundcolor=\color{white},   
    basicstyle=\footnotesize,       
    breakatwhitespace=false,         
    breaklines=true,                 
    captionpos=b,                    
    commentstyle=\color{red},    
    escapeinside={\%*}{*)},          
    extendedchars=true,              
    keepspaces=true,                 
    keywordstyle=\color{blue},       
    language=Python,                
    numbers=left,                    
    numbersep=5pt,                  
    showspaces=false,                
    showstringspaces=false,
    showtabs=false,                  
    tabsize=2
}
\begin{document}
\begin{center}
    \vspace*{1.8cm}
    \Large
    Labwork3\\
\end{center}

\noindent
First, we have an image:
\begin{figure}[H]
\centering
    \includegraphics[height = 0.5\textheight, keepaspectratio]{images/image.jpeg}
    \caption{A simple image}
\end{figure}

\noindent
Then, we reshaped the image to a 2D image and created an input and output buffer on the device.

\begin{lstlisting}[language=Python]
image = cv2.imread("../images/image.jpeg")
image_1d = np.reshape(image, (-1, 3))
height, width, _ = image.shape
pixel_count = height * width
output_device = cuda.device_array((pixel_count,), np.uint8)
input_device = cuda.to_device(image_1d)
\end{lstlisting}

\noindent
First, we defined a function to convert an image to grayscale on the CPU. 

\begin{lstlisting}[language=Python]
def grayscale_cpu(image):
    height = image.shape[0]
    width = image.shape[1]
    image_greyscale = np.zeros((height, width), dtype=np.uint8)
    for h in range(height):
        for w in range(width):
            image_greyscale[h, w] = (
                image[h][w][0] + image[h][w][1] + image[h][w][2]
            ) // 3
    return image_greyscale
\end{lstlisting}

\noindent
The time processed is around 0.30-0.35 seconds. The resulting image is:

\begin{figure}[H]
\centering
    \includegraphics[height = 0.5\textheight, keepaspectratio]{images/grayscale_image_cpu.png}
    \caption{Grayscale image on CPU}
\end{figure}

\noindent
First, we defined a function to convert an image to grayscale on GPU. There are two ways that we can get the thread ID.

\begin{lstlisting}[language=Python]
@cuda.jit
def grayscale_gpu(src, dst):
    x, y = cuda.grid(2)
    # x = cuda.threadIdx.x + cuda.blockIdx.x * cuda.blockDim.x
    # y = cuda.threadIdx.y + cuda.blockIdx.y * cuda.blockDim.y
    if x < dst.shape[0] and y < dst.shape[1]:
        g = np.uint8((src[y, x, 0] + src[y, x, 1] + src[y, x, 2]) // 3)
        dst[x, y] = g
\end{lstlisting}

\noindent
We defined block size and grid size, calculated the time processed, transferred the data from the device to the host, and saved the results.

\begin{lstlisting}[language=Python]
block_size = [(4, 4), (8, 8), (16, 16), (24, 24), (32, 32)]
time_processed_per_block = []
for bs in block_size:
    grid_size_x = math.ceil(width / bs[0])
    grid_size_y = math.ceil(height / bs[1])
    grid_size = (grid_size_x, grid_size_y)
    start = time.time()
    grayscale_gpu[grid_size, bs](input_device, output_device)

    time_processed = time.time() - start
    time_processed_per_block.append(time_processed)

    output_host = output_device.copy_to_host()
    grayscale_image = np.reshape(output_host, (width, height))
    grayscale_image = np.transpose(grayscale_image)

    cv2.imwrite(f"grayscale_image_gpu_{bs}.png", grayscale_image)
\end{lstlisting}

\noindent
The time processed on GPU with different block sizes is:

\begin{lstlisting}[language=Python]
Block size: (4, 4), time processed: 0.19108080863952637
Block size: (8, 8), time processed: 0.00010943412780761719
Block size: (16, 16), time processed: 0.000141143798828125
Block size: (24, 24), time processed: 0.00010418891906738281
Block size: (32, 32), time processed: 0.00010204315185546875
\end{lstlisting}

\noindent
We plot the time comparison between different block sizes. 

\begin{figure}[H]
\centering
    \includegraphics[width = \textwidth, keepaspectratio]{images/comparison.png}
    \caption{Time comparison between different block sizes}
\end{figure}

\noindent
Starting with a block size of (4, 4), the time processed is around 0.18-0.20 seconds, which is already more than 1.5 times faster than CPU processing time. We increased the block size by two and noticed a considerable amount of time decrease, more than 1900 times faster. We further increased the block size; the time slowly decreased, but not as much as before. We noticed that the speed when we used a 2D block size was slower than when we used a 1D block size, around 1.2-1.5 times slower. 



\end{document}
