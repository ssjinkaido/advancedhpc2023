\documentclass[12pt]{article}
\renewcommand{\baselinestretch}{1.2}
\usepackage[utf8]{vietnam}
\usepackage[a4paper, total={6in, 8in}]{geometry}
\usepackage[english]{babel}
\usepackage{hyperref}
\usepackage{mathtools}
\usepackage{amssymb}
\usepackage{indentfirst}
\usepackage{graphicx}
\usepackage{minted}
\usepackage{ragged2e}
\usepackage{multirow}
\usepackage{subcaption}
\usepackage{xurl}
\usepackage{amsmath}
\usepackage{makecell}
\usepackage{algorithm}
\usepackage{algpseudocode}
\renewcommand\theadalign{bc}
\renewcommand\theadfont{\bfseries}
\renewcommand\theadgape{\Gape[4pt]}
\renewcommand\cellgape{\Gape[4pt]}
\usepackage{pbox}
\usepackage{graphicx}
\usepackage{diagbox}
\usepackage{listings}
\usepackage{xcolor}

\hypersetup{
    colorlinks=true,
    linkcolor=blue,
    citecolor=blue,
    urlcolor=blue,
}
\lstset{
    backgroundcolor=\color{white},   
    basicstyle=\footnotesize,       
    breakatwhitespace=false,         
    breaklines=true,                 
    captionpos=b,                    
    commentstyle=\color{red},    
    escapeinside={\%*}{*)},          
    extendedchars=true,              
    keepspaces=true,                 
    keywordstyle=\color{blue},       
    language=Python,                
    numbers=left,                    
    numbersep=5pt,                  
    showspaces=false,                
    showstringspaces=false,
    showtabs=false,                  
    tabsize=2
}
\begin{document}
\begin{center}
    \vspace*{1.8cm}
    \Large
    Labwork6\\
\end{center}

\noindent
First, we have an image:
\begin{figure}[H]
\centering
    \includegraphics[height = 0.5\textheight, keepaspectratio]{images/image.jpeg}
    \caption{A simple image}
\end{figure}


\noindent
First, we defined a function to binarize an image

\begin{lstlisting}[language=Python]
@cuda.jit
def binarize(src, dst, threshold):
    x, y = cuda.grid(2)
    if x < dst.shape[0] and y < dst.shape[1]:
        g = np.uint8((src[y, x, 0] + src[y, x, 1] + src[y, x, 2]) // 3)
        g = 255 if g > threshold else 0
    dst[y, x, 0] = g
    dst[y, x, 1] = g
    dst[y, x, 2] = g
\end{lstlisting}

\noindent
We have the resulting image:
\begin{figure}[H]
\centering
    \includegraphics[height = 0.5\textheight, keepaspectratio]{images/binarize_image.png}
    \caption{Binarized image}
\end{figure}

\noindent
First, we defined a function to increase or decrease the brightness of an image:
\begin{lstlisting}[language=Python]
@cuda.jit
def brightness(src, dst, threshold, brightness_type):
    x, y = cuda.grid(2)
    if x < dst.shape[0] and y < dst.shape[1]:
        if brightness_type == 0:
            b = min(src[y, x, 0] + threshold, 255)
            g = min(src[y, x, 1] + threshold, 255)
            r = min(src[y, x, 2] + threshold, 255)
        elif brightness_type == 1:
            b = max(src[y, x, 0] - threshold, 0)
            g = max(src[y, x, 1] - threshold, 0)
            r = max(src[y, x, 2] - threshold, 0)
    dst[y, x, 0] = b
    dst[y, x, 1] = g
    dst[y, x, 2] = r
\end{lstlisting}

\noindent
We defined a function to call the above function:
\begin{lstlisting}[language=Python]

def show_brightness_image(func, brightness_type):
    for bs in block_size:
        grid_size_x = math.ceil(width / bs[0])
        grid_size_y = math.ceil(height / bs[1])
        grid_size = (grid_size_x, grid_size_y)
        start = time.time()
        func[grid_size, bs](input_device, output_device, threshold, brightness_type)

        time_processed = time.time() - start
        time_processed_per_block.append(time_processed)

        output_host = output_device.copy_to_host()
        brightness_str = "increase" if brightness_type == 0 else "decrease"
        cv2.imwrite(f"{brightness_str}_image.png", output_host)
show_brightness_image(brightness, 0)
show_brightness_image(brightness, 1)
\end{lstlisting}

\noindent
We have the resulting image:
\begin{figure}[H]
\centering
    \includegraphics[height = 0.5\textheight, keepaspectratio]{images/increase_image.png}
    \caption{Increased brightness image}
\end{figure}

\noindent
We have the resulting image:
\begin{figure}[H]
\centering
    \includegraphics[height = 0.5\textheight, keepaspectratio]{images/decrease_image.png}
    \caption{Decreased brightness image}
\end{figure}

\noindent
First, we have another image:
\begin{figure}[H]
\centering
    \includegraphics[height = 0.5\textheight, keepaspectratio]{images/castle.jpeg}
    \caption{A simple image}
\end{figure}

\noindent
First, we defined a function to blend two images:
\begin{lstlisting}[language=Python]
@cuda.jit
def blend(src, src1, dst, c):
    x, y = cuda.grid(2)
    if x < dst.shape[0] and y < dst.shape[1]:
        dst[y, x, 0] = src[y, x, 0] * c + (1 - c) * src1[y, x, 0]
        dst[y, x, 1] = src[y, x, 1] * c + (1 - c) * src1[y, x, 1]
        dst[y, x, 2] = src[y, x, 2] * c + (1 - c) * src1[y, x, 2]
\end{lstlisting}

\noindent
Then, we wrote a function to call it:
\begin{lstlisting}[language=Python]
def blend_image(func):
    for bs in block_size:
        grid_size_x = math.ceil(width / bs[0])
        grid_size_y = math.ceil(height / bs[1])
        grid_size = (grid_size_x, grid_size_y)
        start = time.time()
        func[grid_size, bs](input_device, input_device1, output_device, c)

        time_processed = time.time() - start
        time_processed_per_block.append(time_processed)

        output_host = output_device.copy_to_host()
        cv2.imwrite(f"blend_image.png", output_host)
blend_image(blend)
\end{lstlisting}

\noindent
We have the resulting image:
\begin{figure}[H]
\centering
    \includegraphics[height = 0.5\textheight, keepaspectratio]{images/blend_image.png}
    \caption{Blended image}
\end{figure}


\end{document}
