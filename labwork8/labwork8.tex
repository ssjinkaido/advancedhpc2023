\documentclass[12pt]{article}
\renewcommand{\baselinestretch}{1.2}
\usepackage[utf8]{vietnam}
\usepackage[a4paper, total={6in, 8in}]{geometry}
\usepackage[english]{babel}
\usepackage{hyperref}
\usepackage{mathtools}
\usepackage{amssymb}
\usepackage{indentfirst}
\usepackage{graphicx}
\usepackage{minted}
\usepackage{ragged2e}
\usepackage{multirow}
\usepackage{subcaption}
\usepackage{xurl}
\usepackage{amsmath}
\usepackage{makecell}
\usepackage{algorithm}
\usepackage{algpseudocode}
\renewcommand\theadalign{bc}
\renewcommand\theadfont{\bfseries}
\renewcommand\theadgape{\Gape[4pt]}
\renewcommand\cellgape{\Gape[4pt]}
\usepackage{pbox}
\usepackage{graphicx}
\usepackage{diagbox}
\usepackage{listings}
\usepackage{xcolor}

\hypersetup{
    colorlinks=true,
    linkcolor=blue,
    citecolor=blue,
    urlcolor=blue,
}
\lstset{
    backgroundcolor=\color{white},   
    basicstyle=\footnotesize,       
    breakatwhitespace=false,         
    breaklines=true,                 
    captionpos=b,                    
    commentstyle=\color{red},    
    escapeinside={\%*}{*)},          
    extendedchars=true,              
    keepspaces=true,                 
    keywordstyle=\color{blue},       
    language=Python,                
    numbers=left,                    
    numbersep=5pt,                  
    showspaces=false,                
    showstringspaces=false,
    showtabs=false,                  
    tabsize=2
}
\begin{document}
\begin{center}
    \vspace*{1.8cm}
    \Large
    Labwork8\\
\end{center}

\noindent
First, we have an image:
\begin{figure}[H]
\centering
    \includegraphics[height = 0.5\textheight, keepaspectratio]{images/image.jpeg}
    \caption{A simple image}
\end{figure}


\noindent
First, we defined a function to process an image and convert it from RGB to HSV.

\begin{lstlisting}[language=Python]
image = cv2.imread("../images/image.jpeg")
image = cv2.cvtColor(image, cv2.COLOR_BGR2RGB)
height, width, _ = image.shape
image = np.float32(image)

@cuda.jit
def rgb_to_hsv(src, dst):
    x, y = cuda.grid(2)
    if y < dst.shape[0] and x < dst.shape[1]:
        max_value = max(src[y, x, 0], src[y, x, 1], src[y, x, 2])
        min_value = min(src[y, x, 0], src[y, x, 1], src[y, x, 2])
        delta = max_value - min_value
        if delta == 0:
            h_value = 0
        elif max_value == src[y, x, 0]:
            h_value = 60 * (((src[y, x, 1] - src[y, x, 2]) / delta) % 6)
        elif max_value == src[y, x, 1]:
            h_value = 60 * (((src[y, x, 2] - src[y, x, 0]) / delta) + 2)
        elif max_value == src[y, x, 2]:
            h_value = 60 * (((src[y, x, 0] - src[y, x, 1]) / delta) + 4)

        if max_value == 0:
            s_value = 0
        else:
            s_value = delta / max_value
        v_value = max_value
        dst[y, x, 0] = h_value
        dst[y, x, 1] = s_value
        dst[y, x, 2] = v_value


\end{lstlisting}

\noindent
Then, we defined a function to convert an image from HSV to RGB.
\begin{lstlisting}[language=Python]
@cuda.jit
def hsv_to_rgb(src, dst):
    x, y = cuda.grid(2)

    if y < src.shape[0] and x < src.shape[1]:
        d = src[y, x, 0] / 60
        hi = int(d) % 6
        f = d - hi
        l = src[y, x, 2] * (1 - src[y, x, 1])
        m = src[y, x, 2] * (1 - f * src[y, x, 1])
        n = src[y, x, 2] * (1 - (1 - f) * src[y, x, 1])

        if 0 <= src[y, x, 0] < 60:
            r, g, b = src[y, x, 2], n, l
        elif 60 <= src[y, x, 0] < 120:
            r, g, b = m, src[y, x, 2], l
        elif 120 <= src[y, x, 0] < 180:
            r, g, b = l, src[y, x, 2], n
        elif 180 <= src[y, x, 0] < 240:
            r, g, b = l, m, src[y, x, 2]
        elif 240 <= src[y, x, 0] < 300:
            r, g, b = n, l, src[y, x, 2]
        elif 300 <= src[y, x, 0] < 360:
            r, g, b = src[y, x, 2], l, m

        dst[y, x, 0] = r
        dst[y, x, 1] = g
        dst[y, x, 2] = b

\end{lstlisting}

\noindent
We have the four resulting images. 2 from ours and two using OpenCV built-in functions.

\noindent
We have the resulting image:
\begin{figure}[H]
\centering
    \includegraphics[height = 0.5\textheight, keepaspectratio]{images/hsv_image.png}
    \caption{HSV image from scratch}
\end{figure}

\noindent
We have the resulting image:
\begin{figure}[H]
\centering
    \includegraphics[height = 0.5\textheight, keepaspectratio]{images/hsvcv2_image.png}
    \caption{HSV image OpenCV}
\end{figure}

\noindent
We have the resulting image:
\begin{figure}[H]
\centering
    \includegraphics[height = 0.5\textheight, keepaspectratio]{images/rgb_image.png}
    \caption{RGB image from scratch}
\end{figure}

\noindent
We have the resulting image:
\begin{figure}[H]
\centering
    \includegraphics[height = 0.5\textheight, keepaspectratio]{images/rgbcv2_image.png}
    \caption{RGB image from OpenCV}
\end{figure}


\end{document}
