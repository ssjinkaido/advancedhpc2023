\documentclass[12pt]{article}
\renewcommand{\baselinestretch}{1.2}
\usepackage[utf8]{vietnam}
\usepackage[a4paper, total={6in, 8in}]{geometry}
\usepackage[english]{babel}
\usepackage{hyperref}
\usepackage{mathtools}
\usepackage{amssymb}
\usepackage{indentfirst}
\usepackage{graphicx}
\usepackage{minted}
\usepackage{ragged2e}
\usepackage{multirow}
\usepackage{subcaption}
\usepackage{xurl}
\usepackage{amsmath}
\usepackage{makecell}
\usepackage{algorithm}
\usepackage{algpseudocode}
\renewcommand\theadalign{bc}
\renewcommand\theadfont{\bfseries}
\renewcommand\theadgape{\Gape[4pt]}
\renewcommand\cellgape{\Gape[4pt]}
\usepackage{pbox}
\usepackage{graphicx}
\usepackage{diagbox}
\usepackage{listings}
\usepackage{xcolor}

\hypersetup{
    colorlinks=true,
    linkcolor=blue,
    citecolor=blue,
    urlcolor=blue,
}
\lstset{
    backgroundcolor=\color{white},   
    basicstyle=\footnotesize,       
    breakatwhitespace=false,         
    breaklines=true,                 
    captionpos=b,                    
    commentstyle=\color{red},    
    escapeinside={\%*}{*)},          
    extendedchars=true,              
    keepspaces=true,                 
    keywordstyle=\color{blue},       
    language=Python,                
    numbers=left,                    
    numbersep=5pt,                  
    showspaces=false,                
    showstringspaces=false,
    showtabs=false,                  
    tabsize=2
}
\begin{document}
\begin{center}
    \vspace*{1.8cm}
    \Large
    Labwork9\\
\end{center}

\noindent
First, we have an image:
\begin{figure}[H]
\centering
    \includegraphics[height = 0.5\textheight, keepaspectratio]{images/image.jpeg}
    \caption{A simple image}
\end{figure}


\noindent
First, we defined a function to calculate the histogram of the images

\begin{lstlisting}[language=Python]
@cuda.jit
def histogram(src, hist):
    x, y = cuda.grid(2)
    if y < src.shape[0] and x < src.shape[1]:
        cuda.atomic.add(hist, src[y, x], 1)
\end{lstlisting}

\noindent
Then, we calculated the CDF of the histogram
\begin{lstlisting}[language=Python]
cdf = (np.cumsum(hist) / (width * height)) * 255
\end{lstlisting}

\noindent
We defined the final function to calculate histogram equalization:

\begin{lstlisting}[language=Python]
@cuda.jit
def equalization(src, cdf, dst):
    x, y = cuda.grid(2)
    if y < src.shape[0] and x < src.shape[1]:
        dst[y, x] = cdf[src[y, x]]
\end{lstlisting}
\noindent
We have the resulting image:
\begin{figure}[H]
\centering
    \includegraphics[height = 0.5\textheight, keepaspectratio]{images/he.png}
    \caption{Histogram equalization}
\end{figure}

\end{document}
